\part{Aspectos Gerais}

\chapter[Introdução]{Introdução}




Nas ultimas décadas houve uma evolução muito grande no campo de Engenharia de Software 
e um dos campos que apresentou um grande crescimento foi o da Inteligencia Artificial(IA).
A utilização de IA promoveu diversos avanços nas diferentes áreas dentro da engenharia de software
como na engenharia de requisitos\cite{sofian2022systematic}. A Engenharia de Requisitos (ER) envolve a identificação, análise e documentação de requisitos de software, um processo que é vital para projetos de desenvolvimento \cite{Sommerville2007}.

Um requisito é um atributo necessário em um sistema, uma declaração que determina uma característica, capacidade ou fator de qualidade de um sistema para que ele possua valor e utilidade para um cliente ou usuário. Os requisitos são necessários pois estabelecem a base para todo o trabalho de desenvolvimento subsequente. Uma vez definidos os requisitos, os desenvolvedores dão início às outras atividades técnicas: design do sistema, desenvolvimento, teste, implementação e operação.
A engenharia de requisitos é complexa e envolve mais do que simplesmente registrar as necessidades do cliente. Os requisitos são dinâmicos e mudam ao longo do tempo, à medida que nossa compreensão do problema evolui. É importante ser preciso e flexível ao mesmo tempo. Práticas ineficazes de definição de requisitos são um problema comum na indústria, mas adotando uma abordagem disciplinada é possível aumentar as taxas de sucesso dos projetos\cite{young2004requirements}.

No desenvolvimento de software, a elicitação de requisitos de alta qualidade é um fator crítico para o sucesso. Deficiências na elicitação de requisitos podem impactar negativamente todo o processo de desenvolvimento, resultando em altos custos para as organizações envolvidas. Embora a correção de deficiências durante a fase de elicitação de requisitos tenha sido relatada como responsável por até 75\% de todos os custos de remoção de erros ao longo do ciclo de vida do desenvolvimento de software, a falta de requisitos claros e completos pode até mesmo levar ao fracasso total de um projeto de desenvolvimento de software \cite{urquhart1999themes}.


A adoção da IA na ER é justificada pela necessidade crescente de desenvolver software produtivo, eficiente e de alta qualidade, especialmente no contexto da revolução industrial 4.0 \cite{sofian2022systematic}.
 A IA pode auxiliar na elicitação de requisitos, uma tarefa frequentemente estudada, podendo potencializar os requisitos elicitados nos processos de ER. Além disso, estudos exploratórios recentes têm demonstrado o potencial do aprendizado  de máquina que permite que um modelo faça previsões para classes não vistas durante o treinamento, essa técnica é chamada de aprendizado de máquina não supervisionado. No contexto da Engenharia de Requisitos, essa técnica pode ser utilizada para identificar e classificar requisitos que não foram explicitamente definidos ou considerados durante o desenvolvimento do modelo e apresenta uma nova fronteira de inovação na ER \cite{alhoshan2023zero}. Nessa fase de elicitação de requisitos, a utilização da tecnologia de modelo de aprendizado de máquina não supervisionado como ChatGPT proporciona um desafio em relação a implementação num cenario real. 
 
%  Tendo isto em vista quais os desafios das equipes de desenvolvimento de software ao implementar o uso de tecnologia de modelo de aprendizado de máquina não supervisionado como ChatGPT na fase de elicitação de requisitos e como isso afeta o resultado do processo?

%  qual os desafios do uso de IA para a melhoria no processo de elicitação de requisitos

%  Contudo é necessário avaliar por meio de um estudo de caso real os desafios do uso de IA para melhoria do processo de elicitação de requisitos, 

 Nesse trabalho será apresentado um estudo sobre a utilização dessa tecnologia na etapa de elicitação de requisitos e apresentar um modelo para aplicar em um estudo de caso real.

 \section{Objetivos}
\subsection{Obetivos Gerais}

Fazer uma análise das metodologias de avaliação de desempenho da implementação de IA na engenharia de requisitos e identificar lacunas e oportunidades de melhoria,além da discussão sobre a aplicação e adaptação dessas metodologias para o contexto da elicitação de requisitos. 


\subsection{Objetivos Específicos}

\begin{itemize}
	\item Realizar uma análise crítica das metodologias existentes.
	\item Identificar os principais desafios na engenharia de requisitos no contexto do desenvolvimento de um sistema no Senado.
	\item Propor uma metodologia para avaliar o desempenho e os resultados da implementação da IA na engenharia de requisitos.
	\item Identificar os possíveis impactos do uso da Inteligência Artificial.
\end{itemize}


